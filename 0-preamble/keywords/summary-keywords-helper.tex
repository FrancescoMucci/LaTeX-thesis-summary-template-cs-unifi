%--------------------------------------------------------------
%          Keyword helper commands for thesis summary
%--------------------------------------------------------------

%----------------------KEYWORDS---------------------------------

% The "\keyword{\<name>}{<text>}" command can be used to define a simple keyword: useful to never forget the "\xspace" at the end of the definition (without the "\xspace" every keywords inserted inside text should be followed or enclosed by curly brackets or else there would be no space added).

\newcommand{\keyword}[2]{%
	\newcommand{#1}{#2\xspace}%
}

% For testing
\newcommand{\simpleNewcommand}{newcommand}
\keyword{\keywordExample}{keyword}

%--------------------------------------------------------------

%----------------------EMPHASIZED-KEYWORDS---------------------

% The "\emphKeyword{\<name>}{<text>}" command can be used to define a keyword with emphasized text.

% Basic version.
%\newcommand{\emphKeyword}[2]{%
%	\newcommand{#1}{\emph{#2}\xspace}%
%}

% Improved version.
\newcommand{\emphKeyword}[2]{%
	\keyword{#1}{\emph{#2}}%
}

% For testing.
\emphKeyword{\emphKeywordExample}{emphasized-keyword}

%--------------------------------------------------------------

%----------------------TEXTTT-KEYWORDS-------------------------

% The "\ttKeyword{\<name>}{<text>}" command can be used to define a keyword with typewriter font (useful for code-related keywords).

% Basic version.
%\newcommand{\ttKeyword}[2]{%
%	\newcommand{#1}{\texttt{#2}\xspace}%
%}

% Improved version.
\newcommand{\ttKeyword}[2]{%
	\keyword{#1}{\texttt{#2}}%
}

% For testing.
\ttKeyword{\ttKeywordExample}{typewriter-keyword}

%--------------------------------------------------------------

%------------------HYPERLINKS-HELPER-COMMANDS------------------

% The "\myHref{<URL>}{<text>" command is a wrapper of "\href" from the "hyperref" package (made the text specified as the second argument a hyperlink to the URL specified as the first argument); in my version the text will be rendered using the typewriter font.

\newcommand{\myHref}[2]{%
	\href{#1}{\texttt{#2}}%
}

% The "\https{<URL-without-protocol>}" command can be used to reference a "secure" web page.
\newcommand{\https}[1]{%
	\myHref{https://#1}{#1}%
}

% The "\http{<URL-without-protocol>}" command can be used to reference a "not secure" web page.
\newcommand{\http}[1]{%
	\myHref{http://#1}{#1}%
}

% The "\mailto{<email>}" command can be used to reference an email address.
\newcommand{\mailto}[1]{%
	\myHref{mailto://#1}{#1}%
}

%--------------------------------------------------------------

%----------------------SECURE-WEBPAGE-KEYWORDS-----------------

% The "\webpage{\<name>}{<URL-without-protocol>}" command can be used to define a keyword for a "secure" webpage.

% Basic version.
%\newcommand{\webpage}[2]{%
%	\newcommand{#1}{\href{https://#2}{\texttt{#2}}\xspace}
%}

% Improved version.
\newcommand{\webpage}[2]{%
	\keyword{#1}{\https{#2}}%
}

% For testing.
\webpage{\myWebpage}{francescomucci.github.io}

%--------------------------------------------------------------

%----------------------NOT-SECURE-WEBPAGE-KEYWORDS-------------

% The "\notsecurewebpage{\<name>}{<URL-without-protocol>}" command can be used to define a keyword for a "not secure" webpage.

% Basic version.
%\newcommand{\notsecwebpage}[2]{%
%	\newcommand{#1}{\href{http://#2}{\texttt{#2}}\xspace}
%}

% Improved version.
\newcommand{\notsecwebpage}[2]{%
	\keyword{#1}{\http{#2}}%
}

% For testing.
\notsecwebpage{\notSecureWebpageExample}{icetcs.ru.is}

%--------------------------------------------------------------

%----------------------MAIL-KEYWORDS---------------------------

% The "\mail{\<name>}{<email>}" command can be used to define a keyword for an email address.

% Basic version.
%\newcommand{\mail}[2]{%
%	\newcommand{#1}{\href{mailto://#2}{\texttt{#2}}\xspace}
%}

% Improved version.
\newcommand{\mail}[2]{%
	\keyword{#1}{\mailto{#2}}%
}

% For testing.
\mail{\myTestMail}{francesco.mucci@edu.unifi.it}

%--------------------------------------------------------------

%--------------------SOURCES-FOR-COMMENTS----------------------

% The comments on LaTeX and its commands are based on the contents of https://latexref.xyz/, an unofficial reference manual for the LaTeX2e document preparation system.

% The comments on the classes, styles or packages (and their commands and options) come from the description provided on CTAN (https://www.ctan.org/) and from the official documentation of the different classes, styles or packages.

% The comments on TeX conditional commands are also based on "TeX by Topic" (2017) by Victor Eijkhout.

%---------------------------------------------------------------